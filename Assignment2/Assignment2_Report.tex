\documentclass[11pt]{article}
\usepackage{multicol}
\usepackage{sectsty}
\usepackage{listings}
\usepackage{geometry}
\geometry{
 a4paper,
 total={170mm,257mm},
 left=10mm,
 top=10mm,
}
\usepackage{color}
\definecolor{codegreen}{rgb}{0,0.6,0}
\definecolor{codegray}{rgb}{0.5,0.5,0.5}
\definecolor{codepurple}{rgb}{0.58,0,0.82}

\lstdefinestyle{mystyle}{
    commentstyle=\color{codegreen},
    keywordstyle=\color{magenta},
    numberstyle=\tiny\color{codegray},
    stringstyle=\color{codepurple},
    basicstyle=\footnotesize,
    breakatwhitespace=false,
    breaklines=true,
    captionpos=b,
    keepspaces=true,
    numbers=left,
    numbersep=5pt,
    showspaces=false,
    showstringspaces=false,
    showtabs=false,
    tabsize=2
}

\lstset{style=mystyle}


\title{Assignment 2}
\author{Spiros Mavroidakos 26068931 Jastaj Virdee 260689027}

\begin{document}
\maketitle

\section {Q.1}
\subsection{}
Refer to appendix section A.1

\subsection {}
Yes filter lock does allow other threads to arbitrarily overtake other threads. 
This can be seen by using r-bounded waiting. r-bounded waiting implies that a thread 
(i.e. thread A) cannot overtake another thread (i.e. thread b) more than r times. 
For example, first-come-first-served, has a bounded waiting of r = 0 as it is impossible 
for a thread to overtake another. However, for the filter algorithm, we saw in class that 
there is no value of r which means that a thread can be arbitrarily overtaken.

\subsection {}
Refer to appendix section A.2

\subsection {}
The bakery algorithm does not allow for a thread to overtake another thread an arbitrary number
of times. The bakery algorithm is a first-come-first-served algorithm and therefore has an r = 
0. With r-bounded waiting, an r = 0 means that no thread can overtake another which is by 
definition in a first-come-first-served approach. The bakery algorithms works by a client 
(thread), taking a number and then waiting until all lower numbers have been served. 
This is why a thread cannot overtake another thread an arbitrary number of times.

\subsection {}
A test to verify that mutual exclusion is satisfied is to create an arbitrary amount of 
threads (for testing, a number $>$= 5 will suffice) and have them all try to access the same 
shared resource. However, this resource will be protected using the bakery algorithm. Once 
the lock is obtained, it might be nice to have a print statement declaring that the lock has 
been obtained/released and a sleep statement to make sure that the other threads are not 
accessing it at the same times.

\subsection{}
Refer to appendix section A.3


\appendix
\section{Question 1 Code}
\subsection{Filter algorithm}
\lstinputlisting[language=java]{Filter.java}
\subsection{Bakery Algorithm}
\lstinputlisting[language=java]{Bakery.java}
\subsection{Test Code for Bakery Algorithm}
\lstinputlisting[language=java]{BakeryTest.java}

\end{document}
