\documentclass[11pt]{article}
\usepackage{amsmath,amssymb,amsfonts}
\usepackage{listings}
\usepackage{graphicx}
\graphicspath{ {./images/} }
\usepackage{tikz}
\usepackage{hyperref}
\usepackage{float}
\usepackage{color}
\usepackage{courier}
\usepackage{dirtytalk}
\usepackage{hyperref}
\usepackage{fixltx2e}
\usepackage[margin=1in]{geometry}

\definecolor{dblue}{rgb}{0.2,0.2,0.6}
\definecolor{dyellow}{rgb}{0.6,0.6,0.2}

\lstset{
	basicstyle=\footnotesize\ttfamily,
	breaklines=true,
	showstringspaces=false,
	frame=single,
	showlines=true,
	numbers=left,
}

\lstdefinelanguage{bash}{
	morestring=[b]"",
	morestring=[s]{>}{<},
	stringstyle=\color{black},
	identifierstyle=\color{blue},
	keywordstyle=\color{green},
}

\hypersetup {
	colorlinks = true,
	linkcolor = dblue,
	linktoc = all,
	citecolor = dyellow,
}

\title{Assignment 3}
\author{Spiros-Daniel Mavroidakos - 260689391\\Jastaj Virdee - 260689027}
\date{\today}

\begin{document}
\pagenumbering{gobble}
\maketitle
\pagenumbering{roman}
\newpage
\tableofcontents
\newpage
\pagenumbering{arabic}

\section{Question 1}

\subsection{}
From the question, we know the following.
\begin{itemize}
	\item We are reading L words from cache
	\item Cache Line is 4 words
\end{itemize}
L' is the number of words that can be stored in the cache given a maximum word seperation of stride s. A stride can be a maximum size of 
$\left\lfloor\dfrac{L}{2}\right\rfloor$. We also know that the cache can have several cache lines. We will denote the amount of cache lines 
by k. \\
Knowing this, we can come up with the following equation for L'.

\begin{equation}
	\begin{split}
		L' = \left\lceil\frac{4k}{\left\lfloor\dfrac{L}{2}\right\rfloor}\right\rceil
	\end{split}
\end{equation}

Furthermore, t0 shows the cache access latency as it does take some time to read from the cache. GIven that L < L', the stride size will 
either be 0 or a size that fits within the case. Therefore, all the time delay for access comes from reading the case.

\subsection{}
Given that L > L', the stride will be large enough that we cannot only read from the cache. The words will be too seperated to read only 
from cache. However, at t1, we have peaked in the maximum read time. This implies that the stride is larger than the actual cache size. 
That is why it is a constant line. t1 will therefore be showing the memory access latency since we are no longer reading from the cache. 
t1 might also take into account the time to read the cache as we need to generate a cache miss to read from other storage.

\subsection{}
Part 1 is when L < L' so as stated in section 1.1, all of the data of the array will fit into the cache and will therefore be accessed at 
the time that it takes to read the cache. This is why it is the lowest of the 2 curves. \\
Part 2 is when L > L' however, the stride has not become so big that we cannot use the cache in a meaningful way. Throughout part 2, the 
stride is continuously increasing which is why the curve is increasing as you must read from cache and then main memory. At the end of 
part 2 (the peak value where it becomes constant) is where part 3 begins and when the stride has become so large that the stride takes 
up all of the cache and you will always incur a cache miss and have to then go to main memory which is why it is a peak. Part 2 has a mix 
between some values are in cache and some are not while part 3 is where everything is not in the cache. 

\subsection{}

\end{document}
